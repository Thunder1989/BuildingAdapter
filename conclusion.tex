\section{Conclusion}
We describe a general, simple yet effective feature extraction design in support of sensor type classification with time series data. By experimenting 
with over 2000 streams from two buildings on two campuses, our technique which leverages an ensemble learning method is able to achieve accraucy more than 
92\% and 82\% for testing within building and across buildings respectively. We also discuss that how we make the decision of choosing the window size applied 
to slice the origianl time series and that how the amount of training instances affects the classification accuracy. In general, around 100 instances are 
enough to bootstrap the learning process in the case of 6 types of sensors. Another important piece in the paper is a probability-based solution to identifying 
possibly misclassified instances. With the use of probabilities produced by random forest, in both of the intra- and inter- building learning cases, we are able 
to identify at least 30\% of the misclassifications with an overhead of inspecting some actual correct classification as well.

Our technique can act as a tool in the process of sensor metadata construction. For cases where the type information of sensors is missing, our technique can 
help infer and generate the type metadata. In cases where metadata is avaible in an inconsistent and heterogeneous manner from building to building, our solution 
can be used to verify the type information and unify the naming schema across platforms in different buildings. There still remains the question how abraodly we 
can expand our taxonomy and further study how scalable our technique is before it becomes a generally viable solution to building metadata construciton.
