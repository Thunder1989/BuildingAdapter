\section{Conclusion}
We describe a general, simple yet effective feature extraction design in support of sensor type classification with time series data. By experimenting 
with over 2000 streams from two buildings on two campuses, our technique, which leverages an ensemble learning method, is able to achieve an accuracy more than 
92\% and 82\% for testing within building and across buildings, respectively. We also discuss that how to choose the window size applied 
to a slice of the original time series and how the number of training instances affects classification accuracy. In general, around 100 instances are
enough to bootstrap the learning process in the case of 6 types of sensors. Another important contribution of our paper is a probability-based solution for identifying 
potentially misclassified instances. With the use of probabilities produced by the random forest, in both of the intra- and inter- building learning cases, we are able 
to identify at least 30\% of the misclassifications.% with an overhead of inspecting some actual correct classification as well.

Our technique can act as a tool for metadata construction for building sensors. For cases where type information of sensors is missing, our technique can 
help infer and generate the type metadata. In cases where metadata is available in an inconsistent manner within/across buildings, our solution 
can be used to verify type information and unify the naming schema across platforms in different buildings. Questions remain about how broadly we 
can expand our taxonomy and further study the scalability of our technique.

% is before it becomes a generally viable solution to building metadata construction.
