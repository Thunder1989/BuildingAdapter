\section{Introduction}

To incentivize the reduction of building energy consumption, the U.S. government
launched the Better Buildings Challenge to make buildings at least 20 percent
more efficient by 2020~\cite{doe2013better}. To achieve this goal, many organizations
are applying data analytics to the thousands of sensing and control points\footnote{A sensing or control ``point'' is a sensor, a controller, or a software value.} in
buildings to detect wasteful, incorrect, and unhealthy operation.
There have been many promising analytical approaches~\cite{}, and increasing research attention has been paid to generalize
analytics across multiple buildings as well as integrate multiple analytic engines to gain deeper insights
for identifying the issues in buildings.
However, existing analytical processes are \emph{tightly coupled} to the database schema and the metadata conventions in the buildings for which they were desgined, but the metadata about the type, location, and relationships between the sensing
and control units is not presented consistently across different buildings.
% Such heterogeneity amongst buildings is pervasive and that diversity is reflected in the sensor stock and naming conventions used within and across different buildings.
% Analytical processes require the correspondence between different naming
% conventions, so that points can be mapped to the inputs of these processes.
For instance, if company A wants to execute their
control loops on a building contracted by company B, A needs to manually
map the point names between A and B's system. This process can take weeks and is proportional
to the number of points that need to be used by the application.
Hence, sensor metadata normalization is essential for scaling up the analytical solutions.
% This process is tedious, error-prone, and especially difficult to verify and maintain over time.

In modern commericial buildings, 
% a sensing or control ``point'' is a sensor, a controller, or a software value, e.g., a temperature sensor installed in an office room. The metadata about the point indicates the physical location, the type of sensor or controller, how the sensor or controller relates to the mechanical systems, and other important contextual information. 
% Such information 
metadata is often expressed as short
text strings with several concatenated abbreviations in a point name. Table~\ref{table:ex} lists
a few point names of sensors in three different building management systems
(Trane\footnote{\url{http://www.trane.com/}}, Siemens\footnote{\url{http://www.siemens.com/}}
and Barrington Controls\footnote{The company is no longer in business.}).
For example, the point name \texttt{SODA1R300\_\_ART} is constructed as a
concatenation of the building name (\texttt{SOD}), the air handler unit
identifier (\texttt{A1}), the room number (\texttt{R300}) and the sensor type
(\texttt{ART}, area room temperature). As the name indicates, this point measures
the temperature in a particular room; and it also indicates the control unit that
can affect the temperature in this room. Clearly different naming conventions --
generally guided by the equipment, vendor, manufacturer,
and contractor --
are used in these buildings. For example, the notion of {\em room temperature} is encoded
with a different abbreviation in each of the three buildings: \texttt{Temp}, \texttt{RMT} and \texttt{ART}.
Such variations across different buildings impose great difficulty in quickly deploying automated analytic
solutions.
% \subsection{Metadata Heterogeneity}
\begin{table}[h]
\centering
\begin{tabular}{c|ll}
\cline{1-2}
Building & Point Name & \\
\cline{1-2}
\multirow{2}{*}{\texttt{A}}  & \texttt{Zone Temp 2 RMI204} &  \\
					& \texttt{spaceTemperature 1st Floor Area1} &  \\ \cline{1-2}
\multirow{2}{*}{\texttt{B}} & \texttt{SDH\_SF1\_R282\_RMT} &  \\
                     & \texttt{SDH\_S1-01\_ROOM\_TEMP} &  \\ \cline{1-2}
\multirow{2}{*}{\texttt{C}}  & \texttt{SODA1R300\_\_ART} &  \\
					  & \texttt{SODA1R410B\_ART} &  \\ \cline{1-2}
\end{tabular}
\caption{Example point names for temperature sensors from three different buildings.}
\label{table:ex}
\end{table}

To address this heterogeneity, several solutions have been proposed using a variety of techniques.
Bhattarcharya et al~\cite{arka} use a programming language based solution,
where they derive a set of regular expressions from a handful of labeled examples
to normalize the sensor point names.
%This approach assumes a consistent format for all point names, which is not the case in practice, as shown in Table~\ref{table:ex}.
Schumann et al~\cite{ibm} develop a probabilistic framework to classify sensor types
based on the similarity between a building's point names and entries in a manually constructed dictionary.
%However, the performance of this method is limited by the coverage and diversity of entries listed in the dictionary, and the dictionary size becomes intractable when there exist a lot of variations of the same type, or conflicting definitions of a dictionary entry in different buildings.
Hong et al~\cite{cikm} formulate an active learning based approach to iteratively
acquire human labels for building point names and propagate the acquired labels among similar points in the same building. All prior work is focused on approaches that require manual labeling, which 
% still seriously limits the feasibility of existing analytic solutions at scale. 
continues to be scaling bottleneck;
% \textbf{We need some explanation of this inefficiency.}
even if the sensor data was all publicly available on the Internet, it would still take years if not decades of manual metadata normalization before a new analytics engine could be applied to the 5.6 million commercial buildings in the U.S.


In this paper, we present new techniques to automatically infer metadata in a building {\it without} manual labelling. 
The basic idea is to learn the structure of the streams in a fully normalized building, and transfer that structure to an unnormalized one. 
The insight that guides our solution is that: a sensor typically have two attributes -- its name in text strings and numerical readings generated over time; and these two attributes have different characteristics to be explored for effective knowledge transfer. 
In particular, the names often capture the meaning (e.g., the type) and relationships of streams in a building, but this structure is not easily gleaned from other buildings due to their different naming conventions.
% the name strings have strong regularity and dependency with the sensor types, e.g., in the same building only one or a few abbreviations will be used to denote a particular type of sensors, but such regularity varies across buildings. 
On the other hand, the data contains no such structure, but is more likely to be similar across multiple buildings; the temperature in an office is often very similar, no matter in which building.
% the dependency between the timeseries data and sensor types is more likely to be similar across buildings, but its regularity might be weaker, e.g., the same pattern in the data might be observed in different sensor streams. 
Following this insight, we estimate a set of statistical sensor type classifiers based on the sensor streams in a fully annotated building, transfer them to an unannotated building, and ensemble their output based on the prediction consistency with sensor name strings in the new building. 
To our knowledge, this is the first transfer learning based approach used in this domain.
%Our solution is explicitly designed to \emph{eliminate} manual labeling throughout the normalization process.
% \subsection{Learning from Well-Labeled Buildings}
% Because the sensor types in different buildings
% overlap considerably (e.g., all buildings have temperature sensors installed),
% we can leverage the knowledge about the labels in one building to
% help normalize the labels in another.
% To bridge the gap between naming conventions for different buildings, we envision
% a system that infers the metadata about sensing and control points to construct a normalized
% common standard.
% This would allow faster deployment of analytics/control applications and enable
% building managers to more easily experiment with many different kinds
% of analytic engines and smarter management techniques.

In this work, we take another step towards automatic normalization of building sensor metadata that requires minimal human intervention. 
We focus on a key category of metadata: sensor {\it type} inference, and our technique can automatically infer the type information for a considerable portion for a new building. 
% Fundamentally, transferring type-related ``knowledge'' is achievable because features in the actual timeseries data of streams are common across buildings, even when point names are drastically different.
% For instance, room temperature readings in all buildings are largely in the range of 60-70 Fahrenheit degree while the point names of temperature points can vary substantially, as shown in Table~\ref{table:ex}.
With extensive experimentation, we demonstrate that complementary strengths of both the names and the data 
can be combined to transfer the knowledge from one building to another, leveraging the information 
about the metadata that we learn in each building to improve the model, increase coverage, and maximize performance; the performance can be further boosted as we increase the amount of learning sources.
The labeled subset by our technique could provide a base for other techniques as a warm start.
For example, our technique can complement the traditional labeling technique for a single building.
We summarizes the main contributions in this paper as forllows:
\begin{itemize}\itemsep1pt \parskip1pt \parsep0pt
\item We propose a novel, effective, yet general transfer-learning approach by exploiting domain knowledge learnt from an existing building to help infer information in another building.
\item We empirically analyze how data and the point names of the streams can be transferred for classification across buildings.
\item We evaluate our proposed solution using real data and metadata from over 20 different sensor types and 3,000 sensors in three different commercial buildings.  Our solution can automatically label at least xx\% of the streams with more than yy\% accuracy without human intervention.
\item We further demonstrate how much our transfer learning based approach can accelerate traditional labeling technique for a single building.
\end{itemize}
