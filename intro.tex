\section{Introduction}

To incentivize the reduction of building energy consumption, the U.S. government 
launched the Better Buildings Challenge to make buildings at least 20 percent 
more efficient by 2020~\cite{doe2013better}. To achive this goal, many organizations 
are applying data analytics to the thousands of sensing and control points in 
a typical commercial building to detect wasteful and incorrect operations. These 
complicated analytical processes have proven to be useful and effective on 
a single building, however, an important issue remains that how to make such a process 
generalizable to multiple buildings in a scalable manner. [FIX] need some transition 
before ``however" talking about why we want to port these processes from buildings 
to buildings. On the one hand, analytics-based processes are \emph{tightly coupled} 
to sensors of certain types or locations for each building; on the other hand, 
sensor metadata about the type, location, and relationships between the sensing 
and control units is usually inconsistent across buildings. Therefore, implanting 
these analytical processes requires the correspondence between different naming 
conventions be identified and points be mapped to the inputs of these processes 
before any further analysis is viable. For instance, if Trane wants to execute their 
control loops on a building contracted by Siemens, Trane needs to first manually 
figure out the mapping of the same type of points between Siements' system and 
their own system. This process alone can literally take weeks and is proportional 
to the number of points required in the service. Running the same software across 
buildings requires significant integration effort.


\begin{table}[h]
\centering
\begin{tabular}{c|ll}
\cline{1-2}
Building & Point Name & \\
\cline{1-2}
\multirow{2}{*}{\texttt{A}}  & \texttt{Zone Temp 2 RMI204} &  \\
					& \texttt{spaceTemperature 1st Floor Area1} &  \\ \cline{1-2}
\multirow{2}{*}{\texttt{B}} & \texttt{SDH\_SF1\_R282\_RMT} &  \\
                     & \texttt{SDH\_S1-01\_ROOM\_TEMP} &  \\ \cline{1-2}
\multirow{2}{*}{\texttt{C}}  & \texttt{SODA1R300\_\_ART} &  \\
					  & \texttt{SODA1R410B\_ART} &  \\ \cline{1-2}
\end{tabular}
\caption{Example point names for temperature sensors from three different buildings.}
\label{table:ex}
\end{table}


In modern commercial buildings, a sensing or control ``point'' is a sensor
measurement, a controller, or a software value, e.g., a temperature sensor
installed in an office room. The metadata about the point indicates the physical
location, the type of sensor or controller, how the sensor or controller relates
to the mechanical systems, and other important contextual information. Most of
the time, the metadata is encoded as a ``point name'', which is usually a short
text string with several concatenated abbreviations. Table~\ref{table:ex} lists 
a few point names of sensors in three different building management systems contracted 
by Trane\footnote{\url{http://www.trane.com/}}, Siemens\footnote{\url{http://www.siemens.com/}} 
and Barrington Controls\footnote{The company is no longer in business.}. 
For example, the point name \texttt{SODA1R300\_\_ART} is constructed as a
concatenation of the name of the building (\texttt{SOD}), the air handler unit
identifier (\texttt{A1}), the room number (\texttt{R300}) and the sensor type
(\texttt{ART}, area room temperature). As the name conveys, this point measures 
the temperature in a particular room; and it also indicates the control unit that 
can affect the temperature in this room. Unfortunately, different naming conventions 
are used in most buildings due to different equipment, vendors, manufacturers, 
and contractors being used. As shown in the table, the notion of {\em room temperature} is encoded with a different abbreviation in each of the three buildings: \texttt{Temp}, \texttt{RMT} and \texttt{ART}.
Such variations across different buildings impose great difficulty on scaling up the same analytical process.


Manual normalization of the metadata in buildings becomes intractable due to 
the practical number of buildings. What we often have is one or a few fully 
labeled/normalized buildings. Because the sensor types existing in different buildings 
overlap considerably (e.g., all buildings have temperature sensros installed), it 
would be desirable to leverage the knowledge of one already labeled building to 
help the normalization of a new buidling. 
To abridge the gap between naming conventions for differnt buildings, we envison 
a system that will infer metadata about sensing and control points to a normalized 
common standard.
%based on the knowledge extracted from existing normalized buildings. 
It would allow an advanced analytic engine to quickly connect to and analyze the data 
from a commercial building. Such a system would not only dramatically save time 
but also allow building managers to more easily experiment with many different kinds 
of analytic engines. 
Nevertheless, there is limited work on this topic. 
Bhattarcharya et al~\cite{arka} exploit a programming language based solution, 
where they derive a set of regular expressions from a handful of labeled examples 
to normalize the point name of sensors. 
%This approach assumes a consistent format for all point names, which is not the case in practice, as shown in Table~\ref{table:ex}. 
Schumann et al~\cite{ibm} develop a probabilistic framework to classify sensor types 
based on the similarity of a raw point name to the entries in a manually constructed dictionary. 
%However, the performance of this method is limited by the coverage and diversity of entries listed in the dictionary, and the dictionary size becomes intractable when there exist a lot of variations of the same type, or conflicting definitions of a dictionary entry in different buildings.
Hong et al~\cite{cikm} formulate an active learning based approach to iteratively 
acquire human labels for the building and propogate pseudo labels among points.
However, none of these prior work addresses the scalability issue of metadata 
normalization nor leverages the knowledge from already labeled buildings.

In this work, we demonstrate a first step towards automatic normalization of building 
sensor metadata that requires minimal human intervention. We focus on a key category 
of metadata: the type of a sensing point. In contrast to the aforementioned work, 
where the normalization process is restricted only within the same building, 
we borrow the idea of transfer learning and propose a customized version for transferring 
``knowledge" from an existing building to help infer the sensor type for a new building, which 
effectively minimizes the manual labeling effort throughout the normalization process. 
Based on our observation, the transferral of ``knowledge" is feasible because 
to some extent the actual data of streams carries characteristics that are common 
across buidlings while the point name doesn't. For instance, room temperature 
readings in all buildings should be within the range of 60-70 F degree while 
the point names of temperature points can vary a lot according to 
Table~\ref{table:ex}.

To investigate the effectiveness of the proposed solution for sensor type 
classification across buildings, we performed extensive experiments on a large 
collection of real sensor stream data, which includes over 20 different sensor types 
and 3,000 sensors in three different commercial buildings. To our knowledge, we are 
making the first attempt to tackle this problem systematically and our main 
contributions in this paper are as follows:

\begin{itemize}\itemsep1pt \parskip1pt \parsep0pt
\item We empirically analyze how data and point name of the streams can be transferred for classification across buildings.
\item We propose a novel, effective yet general transfer learning approach by exploiting the domain knowledge learnt from building 1 to help predict on building 2.
\item We evaluate our proposed solution on sensor type classification based on real metadata from over 20 different sensor types and 3,000 sensors in three different commercial buildings; and our solution can automatically label at least xx\% of the streams with more than yy\% accuracy.
\end{itemize}