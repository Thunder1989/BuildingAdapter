\section{Introduction}
Buildings contribute to a large portion of the energy bill in the US.
Buildings are instrumented with sensors to monitor and improve their performance.
The metadata of sensors are usually poorly maintained. We particularly delve into the type information which is useful for a lot of applications such as xx.

We conduct a comprehensive study on the data collected from over 2000 sensors in two separate buildings on two campuses. Our main contributions are:

\begin{itemize}
\item We propose a simple, general yet effective feature extraction scheme to achieve sensor type classification in the context of commercial buildings.
\item We formulate an approach to identifying potential misclassified sensor streams (in terms of the type classes) when no ground truth labels are available.
\item We evaluate our classification technique using data from over 2000 sensor series of 6 types in two buildings on two campuses, and our technique is able to achieve xx\%-xx\% accuracy in intra building scenario and yy\%-yy\% accuracy in inter building scenario.
\item We also evaluate our solution to misclassification identification and the results demonstrate that we are able to identy zz\% percent of the potential population.
\end{itemize}

The rest of the papers is organzied as follows: Section 2 motivates our work. Section 3 gives an overview of the feature extraction design, classification detail as well as misclassification identification. Section 4 presents the results of our evaluation on the proposed techinuqes. Section 5 discusses the limits of our work and some possible future directions. We conclude the paper with Section 6 \& 7.