\section{Introduction}

To incentivize the reduction of building energy consumption, the U.S. government
launched the Better Buildings Challenge to make buildings at least 20 percent
more efficient by 2020~\cite{doe2013better}. To achieve this goal, many
organizations are applying data analytics to the thousands of sensing and
control points\footnote{A sensing or control ``point'' is a sensor, a
  controller, or a software value.} in buildings to detect wasteful, incorrect,
and inefficient operation.  Many promising analytical approaches have been
created that demonstrate promise for substantial energy savings~\cite{find}.
However, these analytic engines are \emph{tightly coupled} to the database
schemas and metadata conventions in the buildings for which they were desgined
and cannot easily be applied to different buildings in which the type, location,
and relationships between are represented differently. Unfortunately, most
buildings use different metadata conventions depending on the type of equipment
in the building, the vendors of the equipment, and the contractors who
originally installed it. The process of {\em mapping} a new building to the
inputs of an analytics engine is currently a manual process that often involves
a technician visiting the building to visually inspect the equipment
installation. It can take days or weeks and is a key obstacle to the widespread
use of building analytics.

Several solutions have been proposed to facilitate the mapping problem.
Bhattarcharya et al~\cite{arka} use a programming language based solution,
where they derive a set of regular expressions from a handful of labeled examples
to normalize the sensor point names.
%This approach assumes a consistent format for all point names, which is not the case in practice, as shown in Table~\ref{table:ex}.
Schumann et al~\cite{ibm} develop a probabilistic framework to classify sensor types
based on the similarity between a building's point names and entries in a manually constructed dictionary.
%However, the performance of this method is limited by the coverage and
%diversity of entries listed in the dictionary, and the dictionary size becomes
%intractable when there exist a lot of variations of the same type, or
%conflicting definitions of a dictionary entry in different buildings.
Hong et al~\cite{cikm} formulate an active learning based approach to
iteratively acquire human labels for building point names and propagate the
acquired labels among similar points in the same building.  These approaches can
significantly reduce the time required for each new building, but they all still
require some manual labeling, one building at a time.

Additionally, the mapping problem cannot be solved simply by throwing more
person hours at the problem. Even after a building is fully mapped, a new
analytics engine may be developed that requires a different kind of metadata,
e.g. which devices are on the northern side of the building, or which sensors
are affected by a given air handler unit. These and other types of metadata may
never even have been encoded in the original databases at all. Thus, as energy
models and building analytics engines become more nuanced, the mapping problem
will become increasingly important.

In this paper, we envision a technology that would enable any new building
analytics engine to quickly be applied to the 10's of millions of commercial
buildings across the globe. Doing so would enable a new market where boutique
analytics could quickly be matched with the buildings they would benefit the
most. The key to this vision will be to infer the building's metadata structure
without any manual intervention at all, removing human attention from the
scalability equation. This end result would not need to be a perfect mapping; it
must only do a first pass that is good enough to flag the right buildings for a
more detailed, manual mapping process.

In this work, we take a first step in the direction of this vision by proposing....

The key insight is that, although they are all
different, there is significant commonality between commercial buildings and the
structure in one building can be used to understand the structure in
another. Even if a building has no exact twin, structure may be infered from a
combination of other buildings.




many

All prior work is focused on approaches that require manual labeling, which 
% still seriously limits the feasibility of existing analytic solutions at scale. 
continues to be scaling bottleneck;
% \textbf{We need some explanation of this inefficiency.}
even if the sensor data was all publicly available on the Internet, it would still take years if not decades of manual metadata normalization before a new analytics engine could be applied to the 5.6 million commercial buildings in the U.S.

In this paper, we present new techniques to automatically infer metadata in a building {\it without} manual labelling. 
The basic idea is to learn the structure of the streams in a fully normalized building, and transfer that structure to an unnormalized one. 
The insight that guides our solution is that: a sensor typically have two attributes -- its name in text strings and numerical readings generated over time; and these two attributes have different characteristics to be explored for effective knowledge transfer. 

In this paper, we present some new techniques to automatically infer metadata in a building {\it without} manual labelling.
% The basic idea is to learn the structure of the streams in a fully normalized building, and transfer that structure to an unnormalized one.
The insight that guides our solution is: a sensor typically has two attributes -- its name in text strings and numerical readings\footnote{We refer to the numerical readings when using the term ``data'' in this paper.} generated over time; and these two attributes have different characteristics to be explored for effective knowledge\footnote{Specifically, ``knowledge'' refers to the predictive pattern extracted from the numerical readings from a building.} transfer.

In particular, the names often capture the meaning (e.g., the type) and relationships of streams in a building, but this structure is not easily gleaned from other buildings due to their different naming conventions.
% the name strings have strong regularity and dependency with the sensor types, e.g., in the same building only one or a few abbreviations will be used to denote a particular type of sensors, but such regularity varies across buildings.
On the other hand, the data contains no such structure, but is more likely to be similar across multiple buildings, e.g., the temperature in an office is often very similar, no matter in which building.
% the dependency between the timeseries data and sensor types is more likely to be similar across buildings, but its regularity might be weaker, e.g., the same pattern in the data might be observed in different sensor streams.
Following this insight, we estimate a set of statistical type classifiers based on the sensor data in a fully annotated building, transfer them to an unannotated building, and ensemble their output based on the prediction consistency with sensor name strings in the new building.
To our knowledge, this is the first transfer learning based approach used in this domain.
%Our solution is explicitly designed to \emph{eliminate} manual labeling throughout the normalization process.
% \subsection{Learning from Well-Labeled Buildings}
% Because the sensor types in different buildings
% overlap considerably (e.g., all buildings have temperature sensors installed),
% we can leverage the knowledge about the labels in one building to
% help normalize the labels in another.
% To bridge the gap between naming conventions for different buildings, we envision
% a system that infers the metadata about sensing and control points to construct a normalized
% common standard.
% This would allow faster deployment of analytics/control applications and enable
% building managers to more easily experiment with many different kinds
% of analytic engines and smarter management techniques.

In this work, we take another step towards automatic normalization of building sensor metadata that requires minimal human intervention.
We focus on a key category of metadata: sensor {\it type} inference, and our technique can automatically infer the type information for a considerable portion of sensing and control points for a new building.
% Fundamentally, transferring type-related ``knowledge'' is achievable because features in the actual timeseries data of streams are common across buildings, even when point names are drastically different.
% For instance, room temperature readings in all buildings are largely in the range of 60-70 Fahrenheit degree while the point names of temperature points can vary substantially, as shown in Table~\ref{table:ex}.
With extensive experimentation, we demonstrate that complementary strengths of both the names and the data
can be combined to transfer the knowledge from one building to another, leveraging the information
about the metadata that we learn in each building to improve the model, increase coverage, and maximize performance; the performance can be further boosted as we increase the amount of learning sources.
The labeled subset by our technique could provide a base for other techniques as a warm start.
For example, our technique can complement the traditional labeling technique for a single building.
We summarizes the main contributions in this paper as follows:
\begin{itemize}\itemsep1pt \parskip1pt \parsep0pt
\item We propose a novel, effective, yet general transfer-learning approach by exploiting domain knowledge learnt from an existing building to help infer information in another building.
\item We empirically analyze how data and the point names of the streams can be transferred for classification across buildings.
\item We evaluate our proposed solution using real data and metadata from over 20 different sensor types and 3,000 sensors in three different commercial buildings.  Our solution can automatically label at least 36\% of the streams with more than 85\% accuracy without human intervention.
\item We further demonstrate how much our transfer learning based approach can accelerate traditional labeling technique for a single building.
\end{itemize}
