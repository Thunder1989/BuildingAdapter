\section{Introduction}
% Buildings contribute to a large portion of the energy bill in the US.
% Buildings are instrumented with sensors to monitor and improve their performance.
% The metadata of sensors are usually poorly maintained. We particularly delve into the type information which is useful for a lot of applications such as xx.

% Buildings consume a large fraction of the energy produced in the US and much of it is wasted.  
% Deep instrumentation is necessary in order to understand
% their dynamics and find opportunities to run them more efficiently.  Moreover, in order to have 
% significant impact across buildings, solutions must
% be widely delivered across a large number of buildings.  
% However, it is well known that building sensor metadata is different across buildings and 
% this places a significant road-block for wide deployment of software-based solutions.

% {\bf points to make: 1) heterogeneity and inconsistency in sensor naming skema. 2) worse case, metadata is misssing, 3) the problem in 1) and 2) make a lot of research efforts hard or impossible to accomplish 4) we can take advantage of both metadata and data available to a) verify, b) generate and c) normalize the metadata.}

Commercial buildings are sites of large sensor/meter deployments used to monitor and optimize their performance.  
With the recent interest in reducing building energy consumption and
increasing their efficiency, it
is important to consider ways to quickly bootstrap a set of building data streams
into an analytical pipeline, such as overall building efficiency or
comfort-assessment analytics and control.
However, because sensor metadata is inconsistent across buildings, software-based
solutions are \emph{tightly coupled} to the sensor metadata conventions (i.e. schemas and naming) for
each building. 
Running the same software across buildings requires significant integration effort.

Current `point' naming conventions and unsystematic recording of metadata form a 
bottleneck in deployment scalability for analytics jobs.  A `point' refers to a 
physical location where a sensor is taking measurements. 
Each building vendor uses their own naming scheme and
unique variants of each scheme are implemented from building to building; variations exist
even across buildings that have contracted the same vendor.
In addition, expanded descriptive information about the point is sometimes unavailable 
-- so determining their meaning is painfully slow or impossible.  
Because these are conventions carried
out by humans, they are inconsistent within and across building data sets.
This makes the integration process laborious for building experts and a non-starter for 
non experts.  The process is fundamentally unscalable.


% Metadata normalization is critical for scaling the deployment process and allows us to \emph{decouple} building-specific conventions 
% from the code written for building applications.  It also allow us to deal with \emph{incorrect} or \emph{missing} metadata.  

% We propose a general, simple, yet effective classification scheme to differentiate sensors in buildings by type.
% We perform ensemble learning on data collected from over 2000 sensor streams in two buildings.
% Our approach is able to achieve more than 92\% accuracy for classification within buildings and more than 82\% accuracy for across buildings. 
% We also introduce a method for identifying potential misclassified streams.  This is important because it allows us to identify opportunities to 
% attain more input from experts -- input that could help improve classification accuracy when ground truth is unavailable.  We show that
% by adjusting a threshold value we are able to identify at least 30\% of the misclassified instances.

% Buildings consume a large fraction of the energy produced in the US and much of it is wasted~\cite{EIA, waste}.
% Many buildings are also sites of very large sensor deployments.  Therefore, in principle, we can observe
% their dynamics and optimize their performance.
% , typically containing
% up to several thousand sensors continuously reporting physical measurement.


% These analyses consist of jobs that measure the performance
% of the building with respect to overall comfort, determine where there are 
% opportunities for energy savings by detecting rooms that drive the energy consumption down --
% also known and rogue rooms/zones -- and finding broken sensors. 
% For broad impact, such analyses should be deployed across many buildings, quickly. 
% However, 

Consider a simple analysis program, which has the ability
to identify anomalous readings from a specific kind of sensor. To execute this job, 
the process organizes each sensor by type and location, organizes a the distribution of
readings across them, and identifies broken sensors where some fraction of
their readings are above some threshold value on the distribution.
The identification step in the process is the most challenging 
because of the problems described.  Ideally, the program would search for points the
way you search for web pages in a search engine -- using semantically meaningful 
terminology. 
% Some codes and metadata across buildings might be unique but we aim 
% to discover the overlap to order for the search results to yield a higher harvest
% (increased coverage of the points that meet the search criteria).

%In order to run this application
%the job needs to know the names of each sensor, its type, and its location.  
Point names contain set of codes that are 
semantically meaningful to the building manager of a specific building.
For example, the point \texttt{BLDA1R435\_\_ART} is constructed as a concatenation of such codes.
The name of the building (first 4 characters), the air handling unit identifier (the 
fifth character), the room number (R435), and the type ART (area room temperature) -- which 
indicates that this are measurement is produced by a temperature sensor. In addition
to point names, there may be some descriptive metadata.  The description for this point (if it 
exists) could describe that this is a ``temperature sensor in room 435''.
However, since point names do not follow the exact same structure within and across
buildings (and certainly do not follow the same convention across vendors)
% Because each point is named by a human, the names can vary. This makes it difficult to construct a general
% set of rules for name construction.  It also means that 
no single approach could solve the normalization problem.  A suite of approaches is necessary.



Metadata normalization is critical for scaling the software deployment process.  
It allows us to
\emph{decouple} building-specific conventions from the code written for building applications.  Normalization allows us to
boost existing metadata, correct incorrect metadata, or generate common metadata when it is missing altogether.
% However, normalization is non-trivial and we believe it requires a suite of solutions. 
One such component in the normalization suite should differentiate sensor feeds by \emph{type}.  For example, we should be able to 
differentiate between sensors measuring 
temperature from sensors measuring pressure.  In addition, we should be able to use what we learn from one building and apply
it to another.  This is especially useful in cases where similar stream types are labeled differently, labeled incorrectly,
 or not labeled at all.

Normalization would allow us to quickly run jobs across many sites by enabling wide \emph{searchability} of points across many buildings at once.
In order to meaningfully deal with disparate building streams in a scalable 
fashion the streams should be \emph{searchable} across various properties, such
as building name, room location, and type.  Moreover, we
assert that wide searchability is necessary for achieving scalability.  By providing a tool for
searching across building streams, we minimize the deployment time for applications;  
allowing them to be used in \emph{all} buildings, not just a single one.

One of the important aspect of the sensor meta/data that we can leverage are the actual patterns in the readings themselves.
Deep inspection of features in the data can yield meaningful results about the \emph{type} of data that it is
and can help us with the label normalization problem.  This paper examines this path using standard machine learning
approaches.  We observe that statistical features over small time windows can be used to identify the stream type.
Moreover, we show that the classification of stream-type can be achieved using an ensemble of classifiers which is
known to outperform a single classifier.

% In this paper we aim to boost the existing 
% metadata by learning the rules of construction through a programming-by-example
% approach where the user provides some input-output examples to boost the existing
% metadata with extra terms.  We can then search across the boosted metadata to increase
% our search harvest over time.

%The aforementioned 
%building management system user interface implicitly groups sensors by location in space
%or association with a system.  This grouping is also captured in the name of the point used by
%the underlying communication protocol.  

%For example, 'AHU' -- air handling unit -- is typically
%embedded in the name of every sensor that is associated with a particular air handling unit.
%A similar convention is used for denoting the type of data produced by the point (i.e. all points
%that contain 'ART' (area room temperature)  in their name refer to a temperature sensor).
%However, these conventions vary slightly across buildings, making it difficult to
%simply integrate based on such tags alone.  We need a way to unify and learn the basic
%set and structure of the tags in order to unify them.

%However, although name conventions are inexact, they 
%are generally similar across buildings with the \emph{same vendor}, so the rules that 
%describe the name in one building might also work to expand the names in another building.
%When automatic construction is not possible or
%uncertain, feedback from the building manager -- the expert who can identify the meaning of 
%more cryptic name encodings -- could provide feedback about the metadata for the sensor.
%We make use of automatic name expansion and programming by example (PBE) 
%techniques using input-output examples, in order to learn
%as much as possible from the set of names available and combine that with expert feedback to
%improve the certainty and coverage of our scheme.

%The stream identification process is manual.  The deployer loads the interface to the 
%building management system, tracks down the spatial or system view, clicks through several windows
%to locate the location of the point(s) of interest, mouses over the point(s) and records the name,
%and then uses that point name to request it from the data-fetch protocol -- typically BACNet or 
%LonTalk or another protocol. This process is repeated in \emph{every building} where this 
%application runs.  Any application that uses building data requires access to the building
%management system and the network carrying the data of interest.

%In addition, not all buildings provide a consistent naming convention sensors within them.
%Therefore, we also explore different ways to expand the metadata in a normalized fashion.
%Commonality across certain statistical features can be used to group
%different streams.  For example, consider the distribution of temperature reading across
%the rooms in the building.  Given a value distribution, it might be easy to pick out values those
%that classify as statistical anoamlies.  If you consider the distribution per sensor type, then
%finding all statistically anomalous sensor might also identify those that are broken.
%This can be determined and indexed a priori, easing the time it takes to identify the 
%streams of interest to applications.

% We propose a set of techniques which learns how to transform a 
% building's metadata 
% to a common namespace by using a small number of examples from an expert. Once the transformation 
% rules are learnt for one building, it can be applied across buildings with a similar 
% metadata structure.  
% We show how our approach makes it easier to write applications across buildings by
% demonstrating its use by three different applications: 1) a rogue zone detector and 
% 2) an application that identifies and ranks the most comfortable
% rooms. We illustrate these on a testbed consisting of nearly 60 buildings comprising more 
% than 16,000 sense points. We also illustrate how this common namespace can help a user write 
% analytics applications that do not require building-specific knowledge and scales across 
% different buildings.

We conduct a comprehensive study on the data collected from over 2000 sensors in two separate buildings on two campuses. Our main contributions are:

\begin{itemize}
\item We propose a simple, general yet effective feature extraction scheme to achieve sensor type classification in the context of commercial buildings.
\item We formulate an approach to identifying potential misclassified sensor streams (in terms of the type classes) when no ground truth labels are available.
\item We evaluate our classification technique using data from over 2000 sensor series of 6 types in two buildings on two campuses, and our technique is able to achieve around 92\% and 98\% accuracy when doing classification within each building, and around 82\% accuracy when inferring type information across buildings.
\item We also evaluate our solution to misclassification identification and the results demonstrate that we are able to identify at least 30\% percent of the target population by choosing an optimal threshold for decision.
\end{itemize}


We believe this is an important study given the recent trends in the penetration
of the \emph{internet of things} into our homes and environments.  Studies show that normalization is an especially pernicious
and widely ubiquitous problem in embedded systems, with only 7\% of data tagged and only 1\% analyzed~\cite{kpcb}.
Our technique can be used to unify that data across many deployment and enable broad search 
and exploration of new applications.  For example, sensing device names for 
the internet of things are likely to follow similar conventions with very little 
context.  We argue that unification through boosting will be necessary in this broader domain.
%Buildings are but one example that serve as a testbed for the proposed techniques.


% We observe that every naming scheme looks to capture three point attributes: 
% 1) the location in space, 2) its relation to an subsytem, and 3) the type of 
% measurement it is taking.  

% We want to make the streams searchable.  How do we do that?
% ) We need index the metadata for the streams but the metadata available is not enough
% 2) We need to expand the metadata, but how?
% 3) name expansion --> tag unification
% 4) timeseries feature extraction --> tag unification

% top things to expand upon:  location, type, system
% secondary: statistical features about the data
 


% The rest of the papers is organized as follows: Section 2 motivates our work. Section 3 details the feature extraction design, classification process as well as a solution to identifying potential misclassified instances. Section 4 discusses the results of our evaluation of the proposed techinuqes. Section 5 discusses the limits of our work and future directions. We conclude the paper with Section 6 \& 7.




