\section{Discussion}
There are several aspects of our work that we left out or did not have time to explore more deeply.
First we go over the expansion of \emph{type} classes and how we could increase coverage of sensor types in future work. 
We discuss how we could improve classification accuracy
by looking for data sources outside the building data sets. We also discuss why principal component analysis is an aspect that we
did not explore in depth and how the principal components can change from building to building.  Finally, we 
explain how our misclassification identifier could be used to improve classification results.

\textbf{Unlabeled Streams}

\textbf{Complementing Traditional Labeling}

\textbf{Better Features for Classifiers}
The performance of transfer learning and classification processes in our work is bounded by the base classifiers which rely only on a set of general features. The line of work to represent time series with discretized symbols (e.g., the SAX~\cite{sax}) or primitive ``shapes'' (e.g., the ``shapelets''~\cite{shapelet1, shapelet2}) doesn't work well for our problem due to the variability in ``shapes'' and unpredictable human noises. There, we wish to explore how using external or domain-specific knowledge could help improve the classification accuracy. 

\textbf{Multiple Buildings as Source}


