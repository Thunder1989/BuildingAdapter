\section{Background and Related Work}
In modern commercial buildings,
% a sensing or control ``point'' is a sensor, a controller, or a software value, e.g., a temperature sensor installed in an office room. The metadata about the point indicates the physical location, the type of sensor or controller, how the sensor or controller relates to the mechanical systems, and other important contextual information.
% Such information
metadata is often expressed as short
text strings with several concatenated abbreviations in a point name. Table~\ref{table:ex} lists
a few point names of sensors in three different building management systems
(Trane\footnote{\url{http://www.trane.com/}}, Siemens\footnote{\url{http://www.siemens.com/}}
and Barrington Controls\footnote{The company is no longer in business.}).
For example, the point name \texttt{SODA1R300\_\_ART} is constructed as a
concatenation of the building name (\texttt{SOD}), the air handler unit
identifier (\texttt{A1}), the room number (\texttt{R300}) and the sensor type
(\texttt{ART}, area room temperature). As the name indicates, this point measures
the temperature in a particular room; and it also indicates the control unit that
can affect the temperature in this room. Clearly different naming conventions --
generally guided by the equipment, vendor, manufacturer,
and contractor --
are used in these buildings. For example, the notion of {\em room temperature} is encoded
with a different abbreviation in each of the three buildings: \texttt{Temp}, \texttt{RMT} and \texttt{ART}.
Such variations across different buildings impose great difficulty in quickly deploying automated analytic
solutions.

% \subsection{Metadata Heterogeneity}
\begin{table}[h]
\centering
\begin{tabular}{c|ll}
\cline{1-2}
Building & Point Name & \\
\cline{1-2}
\multirow{2}{*}{\texttt{A}}  & \texttt{Zone Temp 2 RMI204} &  \\
					& \texttt{spaceTemperature 1st Floor Area1} &  \\ \cline{1-2}
\multirow{2}{*}{\texttt{B}} & \texttt{SDH\_SF1\_R282\_RMT} &  \\
                     & \texttt{SDH\_S1-01\_ROOM\_TEMP} &  \\ \cline{1-2}
\multirow{2}{*}{\texttt{C}}  & \texttt{SODA1R300\_\_ART} &  \\
					  & \texttt{SODA1R410B\_ART} &  \\ \cline{1-2}
\end{tabular}
\caption{Example point names for temperature sensors from three different buildings.}
\label{table:ex}
\end{table}


To the best of our knowledge, we are the first to develop transfer learning based solutions to address the problem of sensor type classification across buildings.

Researchers have tried to systematically address the problem of point name normalization.
Dawson-Haggerty et al.~\cite{boss} and Krioukov et al.~\cite{bas}
introduce a Building Operating System Service stack, whereby
the underlying building sensor stock is presented to applications through a driver-based model \
and an application stack provides a fuzzy-query based interface to the namespace exposed
through the driver interface.
Although this architecture has some useful properties for easing generalizability across
buildings, the driver registration process is still performed manually. Bhattarcharya et al.~\cite{arka} exploit a programming language based solution,
where they derive a set of regular expressions from a handful of labeled examples
to normalize the point name of sensors.
This approach assumes a consistent format for all point names across buildings, which might not be true in practice (as shown in Table \ref{table:ex}).
Schumann et al.~\cite{ibm} develop a probabilistic framework to classify sensor types
based on the similarity of a raw point name to the entries in a manually constructed dictionary.
However, the performance of this method is limited by the coverage and diversity of entries listed in the dictionary, and the dictionary size becomes intractable when there exist a lot of variations of the same type, or conflicting definitions of a dictionary entry in different buildings.
Hong et al~\cite{cikm} formulate an active learning based approach to iteratively
acquire human labels for the target building and propagate pseudo labels among points.
However, all of those existing work depend on manual annotations, and thus none of them address the scalability issue of metadata
normalization across buildings nor leverage the knowledge from already labeled buildings.

Applying transfer learning for cross building sensor type classification saves extra effort in manual annotation by exploiting the labels in the already well annotated buildings.
There are several categories of transfer learning, e.g., inductive, transductive, and multi-task transfer learning as comprehensively surveyed in~\cite{transfer1}.
% We only briefly summarizes the differences.
Inductive transfer learning~\cite{transfer2} assumes the set of class labels in the target domain is different from those in the source domain, and aims at achieving high classification performance in the target domain by transferring knowledge from the source domain. Multi-task transfer learning \cite{multitask} has a similar setting, but tries to learn from the target and source domains simultaneously. Transductive transfer learning~\cite{transfer3} assumes the source and target domains have the same set of labels, but different marginal distribution of features (i.e., $p(x)$) or conditional distribution of labels (i.e., $p(y|x)$). This breaks the basic identical and independent assumption in classical supervised learning models and makes them inept. Typical solutions in transductive transfer learning reweight the source domain trained classifiers' predictions in target domain, e.g., instance-based local weighting~\cite{weight1,weight2,weight3}. But these solutions usually assume that only the marginal distribution of features differ in the source and target domain. Ensemble methods are therefore explored to assign different weights to a set of classifiers to accommodate the varying conditional probabilities of labels in the target domain \cite{ensem1,ensem2}. Our problem setting falls into this category: we assume we have well-labeled instances in one building, but do not have any labeled instances in the target building. We exploit different properties of a sensor point to perform the transfer learning: sensor's  data is utilized to estimate a diverse set of classifiers to transfer knowledge from the source building to target building; sensor names in the target building are used to compute the ensemble weight of classifiers during knowledge transfer.
To the best of our knowledge, we are the first to explore multiple types of features in transfer learning.
% Beyond ensemble-based transfer learning, we explore two types of attributes of a sensor: the features extracted from sensor reading streams are used to estimate a set of different classifiers in source domain; sensor name strings are used to generate clusters in the target domain, and those clusters are then used to assign weights to the source domain trained classifiers when integrating their predictions.
% Such knowledge transfer is possible when the training domain and the test domain have the same set of class labels.


%Timeseries Representation~\cite{sax,shapelet1,shapelet2}

% The use of transfer learning is motivated by the fact that people often have one or a few buildings labeled of which they want to take advantage to aid the labeling of a new buidling.
% Transfer learning is a useful technique in the building space because the effort it takes to label sensors in a single building is high.
% We want to leverage the knowledge gained in one building to quickly label another with minimal effort.
% One category of transfer learning uses well-labeled data from one domain\footnote{A ``domain'' particularly refers to a data set in this paper.}
% to classify examples in a new, related domain.
% Classical supervised learning techniques are not useful for transferring knowledge across domains in this setting because
% The reason that traditional supervised learning techniques is not successful in transferring knowledge across domains in our case is because
% it requires the training and testing examples to be sampled i.i.d. from the same distribution. This basic requirement does not hold here.
