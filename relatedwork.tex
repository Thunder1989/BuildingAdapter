\section{Related Work}
To the best of our knowledge, we are the first to approach the problem of sensor-type classification of physical data in buildings.
  We describe the closest, related work in different problem domains and describe work that uses the random forest as a tool.

There has been much research work on type classification in the context of audio~\cite{audio1,audio2}, music~\cite{music1,music2}, video~\cite{video1,video2}, 
web query~\cite{query1,query2} and human activity~\cite{activity1, activity2}. The goal of~\cite{audio1} is to classify audios into categories such as speech, 
music, background sound and silence using support vector machines, and the work in~\cite{audio2} addresses the same problem with a HMM-based statistical model. 
Examples of music genre (i.e, jazz, pop and so forth) classificatoin are~\cite{music1,music2}, which use GMM with EM algorithm and logistic regression
respectively. And commonly used features for these audio-related classification work are MFCC, zero crossing rate, energy/power and spectral/temporal statistics. 
For video type classification, texture and color-based features are used to classify videos into classes including cartoon, commercial, news and so on 
with decision tree~\cite{video1} and neural network~\cite{video2}. Query categorization has also been researched, ~\cite{query1} exploits a rule-based classifier
while~\cite{query2} uses a Markov random walk model. There is also work on human activity classification in general cases~\cite{activity1} (i.e, running, walking 
and sitting) and home setting~\cite{activity2} (i.e, sleeping, toileting and showering) using accelerometer data with voting-based classifier and HMM with 
conditional random fields respectively. In contrast, our work is focused on sensor type classification using ensemble learning technique.

Random forests have been applied in many different areas~\cite{RF1,RF2,RF3,RF4}. ~\cite{RF1} uses a gene as a feature 
to classify microarray data. ~\cite{RF2} uses the intensity of hundreds of measured metabolites from medical subjects as features, to classify the 
subjects into groups of normal, diseased and diseased with drug treatment. Random forests have also been used in~\cite{RF3} to classify 
 objects in images with image-relevant features. In the area of remote sensing, ~\cite{RF4} utilizes user-defined parameters as features 
 to classify land cover 
types. In our work, we use simple and general statistical feature-set for type classification.

There is work leveraging percentile-based features in time series data for different classification purposes. 
Tarzia~\cite{ABS} 
et. al use a certain percentile in the audio spectrum to classify the current room location. Wang~\cite{business} et. al utilize percentile-based features in audio to characterize occupancy and noise levels. For comparison, we use percentile statistics in sensor time series as 
part of our feature-set to differentiate between different sensor types in commercial buildings.
