\section{Related Work}
To the best of our knowledge, we are the first to approach the problem of sensor type classification leveraging knowledge across buildings.
  We describe the closest, related work in different problem domains and describe work that uses the random forest as a tool.

Bhattarcharya et al~\cite{arka} exploit a programming language based solution, 
where they derive a set of regular expressions from a handful of labeled examples 
to normalize the point name of sensors. 
This approach assumes a consistent format for all point names, which is not the case in practice, as shown in Table~\ref{table:ex}. 
Schumann et al~\cite{ibm} develop a probabilistic framework to classify sensor types 
based on the similarity of a raw point name to the entries in a manually constructed dictionary. 
However, the performance of this method is limited by the coverage and diversity of entries listed in the dictionary, and the dictionary size becomes intractable when there exist a lot of variations of the same type, or conflicting definitions of a dictionary entry in different buildings.
Hong et al~\cite{cikm} formulate an active learning based approach to iteratively 
acquire human labels for the building and propogate pseudo labels among points.
However, none of these prior work addresses the scalability issue of metadata 
normalization nor leverages the knowledge from already labeled buildings.

Tranfer Learning